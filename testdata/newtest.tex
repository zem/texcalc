\documentclass{article}
\usepackage[utf8]{inputenc}
\usepackage{texcalc}
%\usepackage{anfpralayout}
\usepackage{physictools}
\usepackage{ngerman}
\begin{document}

\syserr\usedtexcalc

%\image{texcalc.png}{Latex calculator screenshot}
das istdas satz eins. und hier ist satz zwei.
\ref{texcalc.png} und sartz drei. \ref{another} und vier.
und fübf und sechs.  \\
das istdas satz eins. und hier ist satz zwei.
\ref{texcalc.png} und sartz drei. \ref{another} und vier.
und fübf und sechs.  \\
das istdas satz eins. und hier ist satz zwei.
\ref{texcalc.png} und sartz drei. \ref{another} und vier.
und fübf und sechs.  \\
das istdas satz eins. und hier ist satz zwei.
\ref{texcalc.png} und sartz drei. \ref{another} und vier.
und fübf und sechs.  \\
das istdas satz eins. und hier ist satz zwei.
\ref{texcalc.png} und sartz drei. \ref{another} und vier.
und fübf und sechs.  \\
das istdas satz eins. und hier ist satz zwei.
\ref{texcalc.png} und sartz drei. \ref{another} und vier.
und fübf und sechs.  \\

\limage{texcalc.png}{another Latex calculator screenshot}

das istdas satz eins. und hier ist satz zwei.
\ref{texcalc.png} und sartz drei. \ref{another} und vier.
und fübf und sechs.  \\
das istdas satz eins. und hier ist satz zwei.
\ref{texcalc.png} und sartz drei. \ref{another} und vier.
und fübf und sechs.  \\
das istdas satz eins. und hier ist satz zwei.
\ref{texcalc.png} und sartz drei. \ref{another} und vier.
und fübf und sechs.  \\
das istdas satz eins. und hier ist satz zwei.
\ref{texcalc.png} und sartz drei. \ref{another} und vier.
und fübf und sechs.  \\
das istdas satz eins. und hier ist satz zwei.
\ref{texcalc.png} und sartz drei. \ref{another} und vier.
und fübf und sechs.  \\


das istdas satz eins. und hier ist satz zwei.
\ref{texcalc.png} und sartz drei. \ref{another} und vier.
und fübf und sechs.  \\

\section{NLS}
Locale ist  foo

\begin{calc}
x=f(23.55,0.02)
#r 23.55+/-0.02
def foo(x):
	y=x**2
	return y

z=33
#r 33
foo(z)
#r 1089
y=f(222.0,22)
#r 222.0+/-22.0

a=f(24423.0,220)
#r 24423.0+/-220.0
\end{calc}

$$x_{5sigDigits}=\val[5]{x}$$
$$x_{2sigDigits}=\val{x}$$
$$x^2=\val{foo(x)}$$
$$z=\val{z}$$
$$y=\val{y}$$
$$a=\val{a}$$

\end{document}
