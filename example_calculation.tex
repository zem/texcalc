\documentclass{article}
\usepackage[utf8]{inputenc}
\usepackage{amsmath}
\usepackage{phunivie}
\usepackage{float}
\usepackage{graphicx}
\usepackage{ngerman}
\usepackage{hyperref}

\usepackage{dcolumn}
\newcolumntype{d}{D{,}{,}{2}}

\setlength{\parindent}{0cm}

\begin{document}


%c def grad(g, min, gen=1):
%c 	g=g+0.0
%c 	min=min+0.0
%c 	gen=gen+0.0
%c 	return f((g+(min/60.0), gen/60.0))
%calc

%c def grad2rad(grad):
%c 	return grad*(2*pi/360.0)
%calc

%C phirot=360-grad(321, 19)
%r 38.6833333333+/-0.0166666666667
%C phigelb=360-grad(321, 10)
%r 38.8333333333+/-0.0166666666667
%C phigruen=360-grad(321, 0)
%r 39.0+/-0.0166666666667
%C phiblau=360-grad(320, 21)
%r 39.65+/-0.0166666666667
%C philila=360-grad(320, 0)
%r 40.0+/-0.0166666666667




% Fit 1
%C B=(0.078821756433056|0.00019311935401873)
%r 0.0788217564331+/-0.000193119354019
%C A=-(0.0058425962025917|0.00053680687862343)
%r -0.00584259620259+/-0.000536806878623
%C x=B
%r 0.0788217564331+/-0.000193119354019
%c Iks=B % Kurzschlussstrom
%r 0.0788217564331+/-0.000193119354019
\subsubsection{Kurzschlussstrom}

Für den Kurzschlussstrom wurde ermittelt:

$$I=AU+B=(-0,00584 \pm 0,00054)\cdot U + (0,07882 \pm 0,00020)$$
$$\Rightarrow I_{KS}=B=(0,07882 \pm 0,00020)A$$

% Fit 2
%C B=(0.64389113975977|0.038107608387390)
%r 0.64389113976+/-0.0381076083874
%C A=-(0.69745000936350|0.042148923128185)
%r -0.697450009363+/-0.0421489231282
%C x=(-B)/A
%r 0.923207586372+/-0.0780904776944
%c Ull=x
%r 0.923207586372+/-0.0780904776944

\subsubsection{Leerlaufspannung}

Für die Leerlaufspannung wurde ermittelt:

$$I=AU+B=(-0,697 \pm 0,043)\cdot U + (0,644 \pm 0,039)$$
$$\Rightarrow U_{LL}=\frac{-B}{A}=(0,923 \pm 0,079)V$$

\subsubsection{Maximalleistung}

%C Pmax=(0.05023|0.00001) % W Abgelesen genau
%r 0.05023+/-1e-05
%C Pmax=(0.0502|0.0001) % W Abgelesen
%r 0.0502+/-0.0001
%C Rl=(11.3602|0.0001) % Ohm
%r 11.3602+/-0.0001

$$P_{max}=(0,0502 \pm 0,0001)Watt$$

Der dazugehörige Lastwiderstand: $$R_{L}=(11,3602 \pm 0,0001) \Omega$$


\subsubsection{Kurvenfüllfaktor}
%C CFF=Pmax/(Iks*Ull)
%r 0.68985567969+/-0.0583928092041

Der Kurvenfüllfaktor ergibt sich wie folgt:

$$CFF=\frac{P_{max}}{I_{ks}\cdot U_{LL}}=\frac{(0,0502 \pm 0,0001) W}{(0,07882 \pm 0,00020) A \cdot (0,923 \pm 0,079) V}=(0,690 \pm 0,059)$$


\subsubsection{Widerstand G}

%C gen=0.01
%r 0.01
%C R=f((100, 100*gen)) %Ohm
%r 100.0+/-1.0

Wie in der Anleitung beschrieben wurde zunächst $R_0=(100,0 \pm 1,0)\Omega$ eingestellt und dann der Widerstand gemessen.

%C D=(76|1) %mm
%r 76.0+/-1.0
%C Dr=1000-D % mm
%r 924.0+/-1.0
%C Rx=R*(Dr/D)
%r 1215.78947368+/-21.1554967999

$$R_G=R_0\frac{D_r}{D_l}=(100,0 \pm 1,0) \Omega \cdot \frac{(924,0 \pm 1,0) mm}{(76,0 \pm 1,0) mm} = (1216,0 \pm 22,0) \Omega$$

%Widerstand G:\\
%
%-> bei 100 Ohm\\
%   $76mm \pm 1mm$\\

Die Messung wurde daraufhin mit $R_0=(100,0 \pm 1,0)\Omega$ nocheinmal durchgeführt:
%C R=f((1200, 1200*gen)) %Ohm
%r 1200.0+/-12.0
%C D=(467|1) %mm
%r 467.0+/-1.0
%C Dr=1000-D % mm
%r 533.0+/-1.0
%C Rx=R*(Dr/D)
%r 1369.59314775+/-14.7598879799
$$R_G=R_0\frac{D_r}{D_l}=(1200,0 \pm 12,0) \Omega \cdot \frac{(533,0 \pm 1,0) mm}{(467,0 \pm 1,0) mm} = (1370,0 \pm 15,0) \Omega$$


%-> in der Mitte ($467mm \pm 1mm$)\\
%   1,2 KiloOhm ($= R_0$)\\
\subsubsection{Widerstand M}

%C R=f((6700, 6700*gen)) %Ohm
%r 6700.0+/-67.0
%C D=(491|1) %mm
%r 491.0+/-1.0
%C Dr=1000-D % mm
%r 509.0+/-1.0
%C Rx=R*(Dr/D)
%r 6945.62118126+/-74.8099741312
Der Widerstand wurde mit einem Fluke 183 vermessen um einen
Anhaltspunkt für $R_0=(6700,0 \pm 67,0)\Omega$ zu erhalten.

$$R_M=R_0\frac{D_r}{D_l}=(6700,0 \pm 67,0) \Omega \cdot \frac{(509,0 \pm 1,0) mm}{(491,0 \pm 1,0) mm} = (6946,0 \pm 75,0) \Omega$$

\subsubsection{Widerstand F}

%C R=f((327000, 327000*gen))/1000 %KOhm
%r 327.0+/-3.27
%C D=(501|1) %mm
%r 501.0+/-1.0
%C Dr=1000-D % mm
%r 499.0+/-1.0
%C Rx=R*(Dr/D)
%r 325.694610778+/-3.50784024705
Der Widerstand wurde mit einem Fluke 183 vermessen um einen
Anhaltspunkt für $R_0=(327,0 \pm 3,3)K\Omega$ zu erhalten.

$$R_F=R_0\frac{D_r}{D_l}=(327,0 \pm 3,3) K\Omega \cdot \frac{(499,0 \pm 1,0) mm}{(501,0 \pm 1,0) mm} = (325,7 \pm 3,6) K\Omega$$


\subsection{Diskussion}


%C U0=(1.544|0.001) %V
%r 1.544+/-0.001
Abbildung \ref{IUBatt} zeigt das Verhältnis des gemessenen Stromes zur
Differenz der gemessenen Spannung zur Leerlaufspannung\footnote{Die Leerlaufspannung
wurde im Unbelasteten Zustand der Batterie vor dem Versuch gemessen}
$U_0=(1,5440 \pm 0,0010)V$. Messwerte mit einem Strom von unter $(0,0280\pm 0.0001) A$ wurden für den
Fit ignoriert da diese nicht linear verlaufen. \\

Daraus ergibt sich die Steigung der Kurve und damit ein Innenwiderstand von:

%Nichtlinearer Fit von Datensatz: Tabelle1_Ur, unter Verwendung der Funktion: A*x+B
%Gewichtungsmethode: Keine Gewichtung (alle w_i = 1)
%Skalierter Levenberg-Marquardt Algorithmus mit Tolleranz = 0,0001
%Von x = 2,9850000000000e-02 bis x = 1,3540000000000e-01
%A = 4,5988636959973e-01 +/- 1,0123171017128e-02
%B = 1,2223725911172e-02 +/- 7,9436072948808e-04
%C B=(0.012223725911172|0.00079436072948808)
%r 0.0122237259112+/-0.000794360729488
%C A=(0.45988636959973|0.010123171017128) %Ohm
%r 0.4598863696+/-0.0101231710171
%%C Imin=(0.00766|0.00001) %A Alle messwerte
%C Imin=(0.0280|0.0001) %A
%r 0.028+/-0.0001
%C Imax=(0.1354|0.0001) %A
%r 0.1354+/-0.0001
%C Bmin=B/Imax
%r 0.0902786256364+/-0.00586714939729
%C Bmax=B/Imin
%r 0.436561639685+/-0.0284128372917
%C BI=meanval([nominal_value(Bmin), nominal_value(Bmax)])
%r 0.263420132661+/-0.173141507024
$$R_i=(0,460 \pm 0,011) \Omega$$

\subsection{Diskussion}

Die Ergebnisse entsprechen grundsätzlich den Erwartungen, allerdings wurde beim linearen
Fit die B-Komponente ignoriert, es gilt ja:

$$U_i=AI+B$$
$$\Rightarrow \frac{U_i}{I}=A+\frac{B}{I}$$

Das $\frac{B}{I}$ kann abgeschätzt werden, indem für Minmale und Maximale
Wert gebildet wird, basierend auf den Messwerten die in den Fit eingeflossen sind:

$$B_{min}=\frac{B}{I_{max}}=\frac{(0,01222 \pm 0,00080)}{(0,13540 \pm 0,00010)}=(0,0903 \pm 0,0059)$$
$$B_{max}=\frac{B}{I_{min}}=\frac{(0,01222 \pm 0,00080)}{(0,02800 \pm 0,00010)}=(0,437 \pm 0,029)$$

Durch Bildung des Mittelwertes und der Standarsabweichung erhalten wir:

$$\frac{B}{I}=(0,26 \pm 0,18) \Omega$$

Ein besserer, wenn auch ungenauerer, Wert für den Innenwiderstand wäre demnach:
%C Rimod=A+BI
%r 0.72330650226+/-0.173437193376
$$R_i=A+\frac{B}{I}=(0,460 \pm 0,011)\Omega+(0,26 \pm 0,18)\Omega=(0,72 \pm 0,18)\Omega$$


%C R=10000
%r 10000
%C rlA=(0.12132483338041|0.00014904639696054) %Ohm
%r 0.12132483338+/-0.000149046396961
%C RlA=rlA*R
%r 1213.2483338+/-1.49046396961
%C rlB=(0.33066631086411|0.00024559500833232) %Ohm
%r 0.330666310864+/-0.000245595008332
%C RlB=rlB*R
%r 3306.66310864+/-2.45595008332
%C rlC=(0.68215979518141|0.0010497835935090) %Ohm
%r 0.682159795181+/-0.00104978359351
%C RlC=rlC*R
%r 6821.59795181+/-10.4978359351


\end{document}
\documentclass{article}
\usepackage[utf8]{inputenc}
\usepackage{amsmath}
\usepackage{phunivie}
\usepackage{float}
\usepackage{graphicx}
\usepackage{ngerman}
\usepackage{hyperref}

\usepackage{dcolumn}
\newcolumntype{d}{D{,}{,}{2}}

\setlength{\parindent}{0cm}

\begin{document}


%c def grad(g, min, gen=1):
%c 	g=g+0.0
%c 	min=min+0.0
%c 	gen=gen+0.0
%c 	return f((g+(min/60.0), gen/60.0))
%calc

%c def grad2rad(grad):
%c 	return grad*(2*pi/360.0)
%calc

%C phirot=360-grad(321, 19)
%r 38.6833333333+/-0.0166666666667
%C phigelb=360-grad(321, 10)
%r 38.8333333333+/-0.0166666666667
%C phigruen=360-grad(321, 0)
%r 39.0+/-0.0166666666667
%C phiblau=360-grad(320, 21)
%r 39.65+/-0.0166666666667
%C philila=360-grad(320, 0)
%r 40.0+/-0.0166666666667




% Fit 1
%C B=(0.078821756433056|0.00019311935401873)
%r 0.0788217564331+/-0.000193119354019
%C A=-(0.0058425962025917|0.00053680687862343)
%r -0.00584259620259+/-0.000536806878623
%C x=B
%r 0.0788217564331+/-0.000193119354019
%c Iks=B % Kurzschlussstrom
%r 0.0788217564331+/-0.000193119354019
\subsubsection{Kurzschlussstrom}

Für den Kurzschlussstrom wurde ermittelt:

$$I=AU+B=(-0,00584 \pm 0,00054)\cdot U + (0,07882 \pm 0,00020)$$
$$\Rightarrow I_{KS}=B=(0,07882 \pm 0,00020)A$$

% Fit 2
%C B=(0.64389113975977|0.038107608387390)
%r 0.64389113976+/-0.0381076083874
%C A=-(0.69745000936350|0.042148923128185)
%r -0.697450009363+/-0.0421489231282
%C x=(-B)/A
%r 0.923207586372+/-0.0780904776944
%c Ull=x
%r 0.923207586372+/-0.0780904776944

\subsubsection{Leerlaufspannung}

Für die Leerlaufspannung wurde ermittelt:

$$I=AU+B=(-0,697 \pm 0,043)\cdot U + (0,644 \pm 0,039)$$
$$\Rightarrow U_{LL}=\frac{-B}{A}=(0,923 \pm 0,079)V$$

\subsubsection{Maximalleistung}

%C Pmax=(0.05023|0.00001) % W Abgelesen genau
%r 0.05023+/-1e-05
%C Pmax=(0.0502|0.0001) % W Abgelesen
%r 0.0502+/-0.0001
%C Rl=(11.3602|0.0001) % Ohm
%r 11.3602+/-0.0001

$$P_{max}=(0,0502 \pm 0,0001)Watt$$

Der dazugehörige Lastwiderstand: $$R_{L}=(11,3602 \pm 0,0001) \Omega$$


\subsubsection{Kurvenfüllfaktor}
%C CFF=Pmax/(Iks*Ull)
%r 0.68985567969+/-0.0583928092041

Der Kurvenfüllfaktor ergibt sich wie folgt:

$$CFF=\frac{P_{max}}{I_{ks}\cdot U_{LL}}=\frac{(0,0502 \pm 0,0001) W}{(0,07882 \pm 0,00020) A \cdot (0,923 \pm 0,079) V}=(0,690 \pm 0,059)$$


\subsubsection{Widerstand G}

%C gen=0.01
%r 0.01
%C R=f((100, 100*gen)) %Ohm
%r 100.0+/-1.0

Wie in der Anleitung beschrieben wurde zunächst $R_0=(100,0 \pm 1,0)\Omega$ eingestellt und dann der Widerstand gemessen.

%C D=(76|1) %mm
%r 76.0+/-1.0
%C Dr=1000-D % mm
%r 924.0+/-1.0
%C Rx=R*(Dr/D)
%r 1215.78947368+/-21.1554967999

$$R_G=R_0\frac{D_r}{D_l}=(100,0 \pm 1,0) \Omega \cdot \frac{(924,0 \pm 1,0) mm}{(76,0 \pm 1,0) mm} = (1216,0 \pm 22,0) \Omega$$

%Widerstand G:\\
%
%-> bei 100 Ohm\\
%   $76mm \pm 1mm$\\

Die Messung wurde daraufhin mit $R_0=(100,0 \pm 1,0)\Omega$ nocheinmal durchgeführt:
%C R=f((1200, 1200*gen)) %Ohm
%r 1200.0+/-12.0
%C D=(467|1) %mm
%r 467.0+/-1.0
%C Dr=1000-D % mm
%r 533.0+/-1.0
%C Rx=R*(Dr/D)
%r 1369.59314775+/-14.7598879799
$$R_G=R_0\frac{D_r}{D_l}=(1200,0 \pm 12,0) \Omega \cdot \frac{(533,0 \pm 1,0) mm}{(467,0 \pm 1,0) mm} = (1370,0 \pm 15,0) \Omega$$


%-> in der Mitte ($467mm \pm 1mm$)\\
%   1,2 KiloOhm ($= R_0$)\\
\subsubsection{Widerstand M}

%C R=f((6700, 6700*gen)) %Ohm
%r 6700.0+/-67.0
%C D=(491|1) %mm
%r 491.0+/-1.0
%C Dr=1000-D % mm
%r 509.0+/-1.0
%C Rx=R*(Dr/D)
%r 6945.62118126+/-74.8099741312
Der Widerstand wurde mit einem Fluke 183 vermessen um einen
Anhaltspunkt für $R_0=(6700,0 \pm 67,0)\Omega$ zu erhalten.

$$R_M=R_0\frac{D_r}{D_l}=(6700,0 \pm 67,0) \Omega \cdot \frac{(509,0 \pm 1,0) mm}{(491,0 \pm 1,0) mm} = (6946,0 \pm 75,0) \Omega$$

\subsubsection{Widerstand F}

%C R=f((327000, 327000*gen))/1000 %KOhm
%r 327.0+/-3.27
%C D=(501|1) %mm
%r 501.0+/-1.0
%C Dr=1000-D % mm
%r 499.0+/-1.0
%C Rx=R*(Dr/D)
%r 325.694610778+/-3.50784024705
Der Widerstand wurde mit einem Fluke 183 vermessen um einen
Anhaltspunkt für $R_0=(327,0 \pm 3,3)K\Omega$ zu erhalten.

$$R_F=R_0\frac{D_r}{D_l}=(327,0 \pm 3,3) K\Omega \cdot \frac{(499,0 \pm 1,0) mm}{(501,0 \pm 1,0) mm} = (325,7 \pm 3,6) K\Omega$$


\subsection{Diskussion}


%C U0=(1.544|0.001) %V
%r 1.544+/-0.001
Abbildung \ref{IUBatt} zeigt das Verhältnis des gemessenen Stromes zur
Differenz der gemessenen Spannung zur Leerlaufspannung\footnote{Die Leerlaufspannung
wurde im Unbelasteten Zustand der Batterie vor dem Versuch gemessen}
$U_0=(1,5440 \pm 0,0010)V$. Messwerte mit einem Strom von unter $(0,0280\pm 0.0001) A$ wurden für den
Fit ignoriert da diese nicht linear verlaufen. \\

Daraus ergibt sich die Steigung der Kurve und damit ein Innenwiderstand von:

%Nichtlinearer Fit von Datensatz: Tabelle1_Ur, unter Verwendung der Funktion: A*x+B
%Gewichtungsmethode: Keine Gewichtung (alle w_i = 1)
%Skalierter Levenberg-Marquardt Algorithmus mit Tolleranz = 0,0001
%Von x = 2,9850000000000e-02 bis x = 1,3540000000000e-01
%A = 4,5988636959973e-01 +/- 1,0123171017128e-02
%B = 1,2223725911172e-02 +/- 7,9436072948808e-04
%C B=(0.012223725911172|0.00079436072948808)
%r 0.0122237259112+/-0.000794360729488
%C A=(0.45988636959973|0.010123171017128) %Ohm
%r 0.4598863696+/-0.0101231710171
%%C Imin=(0.00766|0.00001) %A Alle messwerte
%C Imin=(0.0280|0.0001) %A
%r 0.028+/-0.0001
%C Imax=(0.1354|0.0001) %A
%r 0.1354+/-0.0001
%C Bmin=B/Imax
%r 0.0902786256364+/-0.00586714939729
%C Bmax=B/Imin
%r 0.436561639685+/-0.0284128372917
%C BI=meanval([nominal_value(Bmin), nominal_value(Bmax)])
%r 0.263420132661+/-0.173141507024
$$R_i=(0,460 \pm 0,011) \Omega$$

\subsection{Diskussion}

Die Ergebnisse entsprechen grundsätzlich den Erwartungen, allerdings wurde beim linearen
Fit die B-Komponente ignoriert, es gilt ja:

$$U_i=AI+B$$
$$\Rightarrow \frac{U_i}{I}=A+\frac{B}{I}$$

Das $\frac{B}{I}$ kann abgeschätzt werden, indem für Minmale und Maximale
Wert gebildet wird, basierend auf den Messwerten die in den Fit eingeflossen sind:

$$B_{min}=\frac{B}{I_{max}}=\frac{(0,01222 \pm 0,00080)}{(0,13540 \pm 0,00010)}=(0,0903 \pm 0,0059)$$
$$B_{max}=\frac{B}{I_{min}}=\frac{(0,01222 \pm 0,00080)}{(0,02800 \pm 0,00010)}=(0,437 \pm 0,029)$$

Durch Bildung des Mittelwertes und der Standarsabweichung erhalten wir:

$$\frac{B}{I}=(0,26 \pm 0,18) \Omega$$

Ein besserer, wenn auch ungenauerer, Wert für den Innenwiderstand wäre demnach:
%C Rimod=A+BI
%r 0.72330650226+/-0.173437193376
$$R_i=A+\frac{B}{I}=(0,460 \pm 0,011)\Omega+(0,26 \pm 0,18)\Omega=(0,72 \pm 0,18)\Omega$$


%C R=10000
%r 10000
%C rlA=(0.12132483338041|0.00014904639696054) %Ohm
%r 0.12132483338+/-0.000149046396961
%C RlA=rlA*R
%r 1213.2483338+/-1.49046396961
%C rlB=(0.33066631086411|0.00024559500833232) %Ohm
%r 0.330666310864+/-0.000245595008332
%C RlB=rlB*R
%r 3306.66310864+/-2.45595008332
%C rlC=(0.68215979518141|0.0010497835935090) %Ohm
%r 0.682159795181+/-0.00104978359351
%C RlC=rlC*R
%r 6821.59795181+/-10.4978359351


\end{document}
