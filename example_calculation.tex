\documentclass{article}
\usepackage[utf8]{inputenc}
\usepackage{amsmath}
\usepackage{phunivie}
\usepackage{float} 
\usepackage{graphicx}
\usepackage{ngerman}
\usepackage{hyperref}

\usepackage{dcolumn}
\newcolumntype{d}{D{,}{,}{2}}

\setlength{\parindent}{0cm}

\begin{document}


%c def grad(g, min, gen=1):
%c 	g=g+0.0
%c 	min=min+0.0
%c 	gen=gen+0.0
%c 	return f((g+(min/60.0), gen/60.0))
%calc 

%c def grad2rad(grad):
%c 	return grad*(2*pi/360.0)
%calc 

%C phirot=360-grad(321, 19)
%r 38.68333333333334+/-0.016666666666666666
%C phigelb=360-grad(321, 10)
%r 38.833333333333314+/-0.016666666666666666
%C phigruen=360-grad(321, 0)
%r 39.0+/-0.016666666666666666
%C phiblau=360-grad(320, 21)
%r 39.64999999999998+/-0.016666666666666666
%C philila=360-grad(320, 0)
%r 40.0+/-0.016666666666666666




% Fit 1 
%C B=(0.078821756433056|0.00019311935401873)
%r 0.078821756433056+/-0.00019311935401873
%C A=-(0.0058425962025917|0.00053680687862343)
%r -0.0058425962025917+/-0.00053680687862343
%C x=B
%r 0.078821756433056+/-0.00019311935401873
%c Iks=B % Kurzschlussstrom 
\subsubsection{Kurzschlussstrom}

Für den Kurzschlussstrom wurde ermittelt: 

$$I=AU+B=@A@\cdot U + @B@$$
$$\Rightarrow I_{KS}=B=@B@A$$

% Fit 2 
%C B=(0.64389113975977|0.038107608387390)
%r 0.64389113975977+/-0.03810760838739
%C A=-(0.69745000936350|0.042148923128185)
%r -0.6974500093635+/-0.042148923128185
%C x=(-B)/A
%r 0.9232075863722357+/-0.07809047769442526
%c Ull=x

\subsubsection{Leerlaufspannung}

Für die Leerlaufspannung wurde ermittelt: 

$$I=AU+B=@A@\cdot U + @B@$$
$$\Rightarrow U_{LL}=\frac{-B}{A}=@x@V$$
 
\subsubsection{Maximalleistung}

%C Pmax=(0.05023|0.00001) % W Abgelesen genau 
%r 0.05023+/-1e-05
%C Pmax=(0.0502|0.0001) % W Abgelesen
%r 0.0502+/-0.0001
%C Rl=(11.3602|0.0001) % Ohm
%r 11.3602+/-0.0001

$$P_{max}=@Pmax,1@Watt$$

Der dazugehörige Lastwiderstand: $$R_{L}=@Rl,1@ \Omega$$


\subsubsection{Kurvenfüllfaktor}
%C CFF=Pmax/(Iks*Ull)
%r 0.6898556796903881+/-0.05839280920411855

Der Kurvenfüllfaktor ergibt sich wie folgt: 

$$CFF=\frac{P_{max}}{I_{ks}\cdot U_{LL}}=\frac{@Pmax,1@ W}{@Iks@ A \cdot @Ull@ V}=@CFF@$$


\subsubsection{Widerstand G}

%C gen=0.01
%r 0.01
%C R=f((100, 100*gen)) %Ohm 
%r 100.0+/-1.0

Wie in der Anleitung beschrieben wurde zunächst $R_0=@R,1@\Omega$ eingestellt und dann der Widerstand gemessen. 

%C D=(76|1) %mm
%r 76.0+/-1.0
%C Dr=1000-D % mm
%r 924.0+/-1.0
%C Rx=R*(Dr/D)
%r 1215.7894736842104+/-21.15549679990502

$$R_G=R_0\frac{D_r}{D_l}=@R,1@ \Omega \cdot \frac{@Dr,0@ mm}{@D,0@ mm} = @Rx@ \Omega$$

%Widerstand G:\\
%
%-> bei 100 Ohm\\
%   $76mm \pm 1mm$\\

Die Messung wurde daraufhin mit $R_0=@R,1@\Omega$ nocheinmal durchgeführt: 
%C R=f((1200, 1200*gen)) %Ohm 
%r 1200.0+/-12.0
%C D=(467|1) %mm
%r 467.0+/-1.0
%C Dr=1000-D % mm
%r 533.0+/-1.0
%C Rx=R*(Dr/D)
%r 1369.593147751606+/-14.759887979871142
$$R_G=R_0\frac{D_r}{D_l}=@R,1@ \Omega \cdot \frac{@Dr,0@ mm}{@D,0@ mm} = @Rx@ \Omega$$


%-> in der Mitte ($467mm \pm 1mm$)\\
%   1,2 KiloOhm ($= R_0$)\\
\subsubsection{Widerstand M}

%C R=f((6700, 6700*gen)) %Ohm 
%r 6700.0+/-67.0
%C D=(491|1) %mm
%r 491.0+/-1.0
%C Dr=1000-D % mm
%r 509.0+/-1.0
%C Rx=R*(Dr/D)
%r 6945.621181262729+/-74.80997413119528
Der Widerstand wurde mit einem Fluke 183 vermessen um einen 
Anhaltspunkt für $R_0=@R@\Omega$ zu erhalten.

$$R_M=R_0\frac{D_r}{D_l}=@R,1@ \Omega \cdot \frac{@Dr,0@ mm}{@D,0@ mm} = @Rx@ \Omega$$

\subsubsection{Widerstand F}

%C R=f((327000, 327000*gen))/1000 %KOhm 
%r 327.0+/-3.27
%C D=(501|1) %mm
%r 501.0+/-1.0
%C Dr=1000-D % mm
%r 499.0+/-1.0
%C Rx=R*(Dr/D)
%r 325.6946107784431+/-3.507840247045031
Der Widerstand wurde mit einem Fluke 183 vermessen um einen 
Anhaltspunkt für $R_0=@R@K\Omega$ zu erhalten.

$$R_F=R_0\frac{D_r}{D_l}=@R@ K\Omega \cdot \frac{@Dr,0@ mm}{@D,0@ mm} = @Rx@ K\Omega$$


\subsection{Diskussion}


%C U0=(1.544|0.001) %V
%r 1.544+/-0.001
Abbildung \ref{IUBatt} zeigt das Verhältnis des gemessenen Stromes zur
Differenz der gemessenen Spannung zur Leerlaufspannung\footnote{Die Leerlaufspannung 
wurde im Unbelasteten Zustand der Batterie vor dem Versuch gemessen} 
$U_0=@U0@V$. Messwerte mit einem Strom von unter $(0,0280\pm 0.0001) A$ wurden für den 
Fit ignoriert da diese nicht linear verlaufen. \\

Daraus ergibt sich die Steigung der Kurve und damit ein Innenwiderstand von: 

%Nichtlinearer Fit von Datensatz: Tabelle1_Ur, unter Verwendung der Funktion: A*x+B
%Gewichtungsmethode: Keine Gewichtung (alle w_i = 1)
%Skalierter Levenberg-Marquardt Algorithmus mit Tolleranz = 0,0001
%Von x = 2,9850000000000e-02 bis x = 1,3540000000000e-01
%A = 4,5988636959973e-01 +/- 1,0123171017128e-02
%B = 1,2223725911172e-02 +/- 7,9436072948808e-04
%C B=(0.012223725911172|0.00079436072948808)
%r 0.012223725911172+/-0.00079436072948808
%C A=(0.45988636959973|0.010123171017128) %Ohm 
%r 0.45988636959973+/-0.010123171017128
%%C Imin=(0.00766|0.00001) %A Alle messwerte 
%C Imin=(0.0280|0.0001) %A
%r 0.028+/-0.0001
%C Imax=(0.1354|0.0001) %A
%r 0.1354+/-0.0001
%C Bmin=B/Imax 
%r 0.09027862563642541+/-0.005867149397288685
%C Bmax=B/Imin
%r 0.4365616396847143+/-0.028412837291722805
%C BI=meanval([nominal_value(Bmin), nominal_value(Bmax)]) 
%r 0.26342013266056985+/-0.17314150702414444
$$R_i=@A@ \Omega$$

\subsection{Diskussion}

Die Ergebnisse entsprechen grundsätzlich den Erwartungen, allerdings wurde beim linearen 
Fit die B-Komponente ignoriert, es gilt ja: 

$$U_i=AI+B$$
$$\Rightarrow \frac{U_i}{I}=A+\frac{B}{I}$$

Das $\frac{B}{I}$ kann abgeschätzt werden, indem für Minmale und Maximale 
Wert gebildet wird, basierend auf den Messwerten die in den Fit eingeflossen sind: 

$$B_{min}=\frac{B}{I_{max}}=\frac{@B@}{@Imax@}=@Bmin@$$
$$B_{max}=\frac{B}{I_{min}}=\frac{@B@}{@Imin@}=@Bmax@$$

Durch Bildung des Mittelwertes und der Standarsabweichung erhalten wir: 

$$\frac{B}{I}=@BI@ \Omega$$

Ein besserer, wenn auch ungenauerer, Wert für den Innenwiderstand wäre demnach: 
%C Rimod=A+BI
%r 0.7233065022602998+/-0.17343719337568247
$$R_i=A+\frac{B}{I}=@A@\Omega+@BI@\Omega=@Rimod@\Omega$$


%C R=10000 
%r 10000
%C rlA=(0.12132483338041|0.00014904639696054) %Ohm 
%r 0.12132483338041+/-0.00014904639696054
%C RlA=rlA*R
%r 1213.2483338041+/-1.4904639696054
%C rlB=(0.33066631086411|0.00024559500833232) %Ohm 
%r 0.33066631086411+/-0.00024559500833232
%C RlB=rlB*R
%r 3306.6631086411003+/-2.4559500833232
%C rlC=(0.68215979518141|0.0010497835935090) %Ohm 
%r 0.68215979518141+/-0.001049783593509
%C RlC=rlC*R
%r 6821.5979518141+/-10.49783593509


\end{document}
\documentclass{article}
\usepackage[utf8]{inputenc}
\usepackage{amsmath}
\usepackage{phunivie}
\usepackage{float} 
\usepackage{graphicx}
\usepackage{ngerman}
\usepackage{hyperref}

\usepackage{dcolumn}
\newcolumntype{d}{D{,}{,}{2}}

\setlength{\parindent}{0cm}

\begin{document}


%c def grad(g, min, gen=1):
%c 	g=g+0.0
%c 	min=min+0.0
%c 	gen=gen+0.0
%c 	return f((g+(min/60.0), gen/60.0))
%calc 

%c def grad2rad(grad):
%c 	return grad*(2*pi/360.0)
%calc 

%C phirot=360-grad(321, 19)
%r 38.68333333333334+/-0.016666666666666666
%C phigelb=360-grad(321, 10)
%r 38.833333333333314+/-0.016666666666666666
%C phigruen=360-grad(321, 0)
%r 39.0+/-0.016666666666666666
%C phiblau=360-grad(320, 21)
%r 39.64999999999998+/-0.016666666666666666
%C philila=360-grad(320, 0)
%r 40.0+/-0.016666666666666666




% Fit 1 
%C B=(0.078821756433056|0.00019311935401873)
%r 0.078821756433056+/-0.00019311935401873
%C A=-(0.0058425962025917|0.00053680687862343)
%r -0.0058425962025917+/-0.00053680687862343
%C x=B
%r 0.078821756433056+/-0.00019311935401873
%c Iks=B % Kurzschlussstrom 
\subsubsection{Kurzschlussstrom}

Für den Kurzschlussstrom wurde ermittelt: 

$$I=AU+B=@A@\cdot U + @B@$$
$$\Rightarrow I_{KS}=B=@B@A$$

% Fit 2 
%C B=(0.64389113975977|0.038107608387390)
%r 0.64389113975977+/-0.03810760838739
%C A=-(0.69745000936350|0.042148923128185)
%r -0.6974500093635+/-0.042148923128185
%C x=(-B)/A
%r 0.9232075863722357+/-0.07809047769442526
%c Ull=x

\subsubsection{Leerlaufspannung}

Für die Leerlaufspannung wurde ermittelt: 

$$I=AU+B=@A@\cdot U + @B@$$
$$\Rightarrow U_{LL}=\frac{-B}{A}=@x@V$$
 
\subsubsection{Maximalleistung}

%C Pmax=(0.05023|0.00001) % W Abgelesen genau 
%r 0.05023+/-1e-05
%C Pmax=(0.0502|0.0001) % W Abgelesen
%r 0.0502+/-0.0001
%C Rl=(11.3602|0.0001) % Ohm
%r 11.3602+/-0.0001

$$P_{max}=@Pmax,1@Watt$$

Der dazugehörige Lastwiderstand: $$R_{L}=@Rl,1@ \Omega$$


\subsubsection{Kurvenfüllfaktor}
%C CFF=Pmax/(Iks*Ull)
%r 0.6898556796903881+/-0.05839280920411855

Der Kurvenfüllfaktor ergibt sich wie folgt: 

$$CFF=\frac{P_{max}}{I_{ks}\cdot U_{LL}}=\frac{@Pmax,1@ W}{@Iks@ A \cdot @Ull@ V}=@CFF@$$


\subsubsection{Widerstand G}

%C gen=0.01
%r 0.01
%C R=f((100, 100*gen)) %Ohm 
%r 100.0+/-1.0

Wie in der Anleitung beschrieben wurde zunächst $R_0=@R,1@\Omega$ eingestellt und dann der Widerstand gemessen. 

%C D=(76|1) %mm
%r 76.0+/-1.0
%C Dr=1000-D % mm
%r 924.0+/-1.0
%C Rx=R*(Dr/D)
%r 1215.7894736842104+/-21.15549679990502

$$R_G=R_0\frac{D_r}{D_l}=@R,1@ \Omega \cdot \frac{@Dr,0@ mm}{@D,0@ mm} = @Rx@ \Omega$$

%Widerstand G:\\
%
%-> bei 100 Ohm\\
%   $76mm \pm 1mm$\\

Die Messung wurde daraufhin mit $R_0=@R,1@\Omega$ nocheinmal durchgeführt: 
%C R=f((1200, 1200*gen)) %Ohm 
%r 1200.0+/-12.0
%C D=(467|1) %mm
%r 467.0+/-1.0
%C Dr=1000-D % mm
%r 533.0+/-1.0
%C Rx=R*(Dr/D)
%r 1369.593147751606+/-14.759887979871142
$$R_G=R_0\frac{D_r}{D_l}=@R,1@ \Omega \cdot \frac{@Dr,0@ mm}{@D,0@ mm} = @Rx@ \Omega$$


%-> in der Mitte ($467mm \pm 1mm$)\\
%   1,2 KiloOhm ($= R_0$)\\
\subsubsection{Widerstand M}

%C R=f((6700, 6700*gen)) %Ohm 
%r 6700.0+/-67.0
%C D=(491|1) %mm
%r 491.0+/-1.0
%C Dr=1000-D % mm
%r 509.0+/-1.0
%C Rx=R*(Dr/D)
%r 6945.621181262729+/-74.80997413119528
Der Widerstand wurde mit einem Fluke 183 vermessen um einen 
Anhaltspunkt für $R_0=@R@\Omega$ zu erhalten.

$$R_M=R_0\frac{D_r}{D_l}=@R,1@ \Omega \cdot \frac{@Dr,0@ mm}{@D,0@ mm} = @Rx@ \Omega$$

\subsubsection{Widerstand F}

%C R=f((327000, 327000*gen))/1000 %KOhm 
%r 327.0+/-3.27
%C D=(501|1) %mm
%r 501.0+/-1.0
%C Dr=1000-D % mm
%r 499.0+/-1.0
%C Rx=R*(Dr/D)
%r 325.6946107784431+/-3.507840247045031
Der Widerstand wurde mit einem Fluke 183 vermessen um einen 
Anhaltspunkt für $R_0=@R@K\Omega$ zu erhalten.

$$R_F=R_0\frac{D_r}{D_l}=@R@ K\Omega \cdot \frac{@Dr,0@ mm}{@D,0@ mm} = @Rx@ K\Omega$$


\subsection{Diskussion}


%C U0=(1.544|0.001) %V
%r 1.544+/-0.001
Abbildung \ref{IUBatt} zeigt das Verhältnis des gemessenen Stromes zur
Differenz der gemessenen Spannung zur Leerlaufspannung\footnote{Die Leerlaufspannung 
wurde im Unbelasteten Zustand der Batterie vor dem Versuch gemessen} 
$U_0=@U0@V$. Messwerte mit einem Strom von unter $(0,0280\pm 0.0001) A$ wurden für den 
Fit ignoriert da diese nicht linear verlaufen. \\

Daraus ergibt sich die Steigung der Kurve und damit ein Innenwiderstand von: 

%Nichtlinearer Fit von Datensatz: Tabelle1_Ur, unter Verwendung der Funktion: A*x+B
%Gewichtungsmethode: Keine Gewichtung (alle w_i = 1)
%Skalierter Levenberg-Marquardt Algorithmus mit Tolleranz = 0,0001
%Von x = 2,9850000000000e-02 bis x = 1,3540000000000e-01
%A = 4,5988636959973e-01 +/- 1,0123171017128e-02
%B = 1,2223725911172e-02 +/- 7,9436072948808e-04
%C B=(0.012223725911172|0.00079436072948808)
%r 0.012223725911172+/-0.00079436072948808
%C A=(0.45988636959973|0.010123171017128) %Ohm 
%r 0.45988636959973+/-0.010123171017128
%%C Imin=(0.00766|0.00001) %A Alle messwerte 
%C Imin=(0.0280|0.0001) %A
%r 0.028+/-0.0001
%C Imax=(0.1354|0.0001) %A
%r 0.1354+/-0.0001
%C Bmin=B/Imax 
%r 0.09027862563642541+/-0.005867149397288685
%C Bmax=B/Imin
%r 0.4365616396847143+/-0.028412837291722805
%C BI=meanval([nominal_value(Bmin), nominal_value(Bmax)]) 
%r 0.26342013266056985+/-0.17314150702414444
$$R_i=@A@ \Omega$$

\subsection{Diskussion}

Die Ergebnisse entsprechen grundsätzlich den Erwartungen, allerdings wurde beim linearen 
Fit die B-Komponente ignoriert, es gilt ja: 

$$U_i=AI+B$$
$$\Rightarrow \frac{U_i}{I}=A+\frac{B}{I}$$

Das $\frac{B}{I}$ kann abgeschätzt werden, indem für Minmale und Maximale 
Wert gebildet wird, basierend auf den Messwerten die in den Fit eingeflossen sind: 

$$B_{min}=\frac{B}{I_{max}}=\frac{@B@}{@Imax@}=@Bmin@$$
$$B_{max}=\frac{B}{I_{min}}=\frac{@B@}{@Imin@}=@Bmax@$$

Durch Bildung des Mittelwertes und der Standarsabweichung erhalten wir: 

$$\frac{B}{I}=@BI@ \Omega$$

Ein besserer, wenn auch ungenauerer, Wert für den Innenwiderstand wäre demnach: 
%C Rimod=A+BI
%r 0.7233065022602998+/-0.17343719337568247
$$R_i=A+\frac{B}{I}=@A@\Omega+@BI@\Omega=@Rimod@\Omega$$


%C R=10000 
%r 10000
%C rlA=(0.12132483338041|0.00014904639696054) %Ohm 
%r 0.12132483338041+/-0.00014904639696054
%C RlA=rlA*R
%r 1213.2483338041+/-1.4904639696054
%C rlB=(0.33066631086411|0.00024559500833232) %Ohm 
%r 0.33066631086411+/-0.00024559500833232
%C RlB=rlB*R
%r 3306.6631086411003+/-2.4559500833232
%C rlC=(0.68215979518141|0.0010497835935090) %Ohm 
%r 0.68215979518141+/-0.001049783593509
%C RlC=rlC*R
%r 6821.5979518141+/-10.49783593509


\end{document}
