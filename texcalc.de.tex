\documentclass[a4paper]{article}
\usepackage[utf8]{inputenc}
\usepackage{german}
\usepackage{anfpralayout}
\usepackage{physictools}
\usepackage{texcalc}
\usepackage{amsmath}

\setlength{\parindent}{0cm}

\begin{document}
\title{Anleitung zu den {\TeX}calc \LaTeXe-Packeten}
\author{Hans Freitag (und einige andere)}
\maketitle 


Der Packetname {\bf phunivie} steht für Fakultät für Physik der Uni Wien. 
Das Packet ist eine Sammlung von \LaTeXe-Makros von Studierenden der Fakultät, 
die sich im Laufe des Studiums als Hilfreich erwiesen haben. 

\tableofcontents

\section{Features}


\section{Installation}

Das Packet ist in der Datei {\bf phunivie.sty} implementiert, um das Packet zu 
verwenden muss es entweder in dem Verzeichnis liegen wo das \TeX-Dokument 
übersetzt wird, oder in eines der Suchpfade für Packete kopiert werden.\\

\subsection{Linux}

Unter Linux kann man das Packet unter  {\bf
\$HOME/texmf/tex/latex/phunivie/phunivie.sty} ablegen, dann wird es von allen 
Dokumenten gefunden. \\

\subsection{Windows}
\begin{enumerate}
	\item Erstellen eines Ordners mit dem Namen der \textbf{\textbf{*.sty}} 
		Datei (hier phunivie) unter dem Installationsordner:\\

		\textbf{\emph{MiKTeX\textbackslash tex\textbackslash latex}}\\
		Normalerweise unter: \textbf{\emph{C:\textbackslash Program 
		Files (x86)\textbackslash MiKTeX 2.9\textbackslash 
		tex\textbackslash latex}}

	\item Einbinden in Miktex:\\
			Gehe auf $\boldsymbol{Start \rightarrow MiKTeX \rightarrow 
			Maintenance \rightarrow Settings \rightarrow Refresh~FNDB}$

	\item Fertig!
\end{enumerate}


\section{Benutzung}

In der Preambel:

\begin{verbatim}
	\usepackage{phunivie}
\end{verbatim}

oder

\begin{verbatim}
	\usepackage[option]{phunivie}
\end{verbatim}

Optionen können sein: 

\begin{description}
	\item[basti] Sebastians Layout Parameter (margin$=$2.5cm)
	\item[motz] Motz's Layout Parameter
\end{description}
	

\section{Titelseite}

Für das Anfängerpraktikum ist eine Besondere Titelseite nötig, die 
kann natürlich ganz einfach gesetzt werden:

\begin{verbatim}
\semester{SoSe 2011}
%\fakulty{Fakultät für Sonstwas}
%\title{Physikalisches Praktikum\\für Lehramtskandidaten}
\experiment{1. Fnord}
%\date{abgabedatum}
\author{Hans Freitag}
\group{3}
\supervisor{someone}

\makeanfpratitle
\end{verbatim}

Die auskommentierten Parameter sind optional. Und so siehts 
aus\footnote{$\backslash$date und $\backslash$author gehen zur Zeit nur wenn der Block am 
Anfang eines Dokumentes steht}:

\semester{SoSe 2011}
%\fakulty{Fakultät für Sonstwas}
%\title{Physikalisches Praktikum\\für Lehramtskandidaten}
\experiment{1. Fnord}
%\date{\today}
%\author{Hans Freitag}
\group{3}
\supervisor{someone}
\makeanfpratitle


\section{physictools Packet}


\subsection{$23\E{10}$}

\begin{verbatim}
$$23\E{10}$$
\end{verbatim}

\subsection{\ddt}

\begin{verbatim}
$$
	\ddt
$$
\end{verbatim}

\subsection{$\dnach{x}$}

\begin{verbatim}
$$
	\dnach{x}
$$
\end{verbatim}

\subsection{$\ddnach{x}{y}$}

\begin{verbatim}
$$
	\ddnach{x}{y}
$$
\end{verbatim}


\subsection{$\dddnach{x}{y}{z}$}

\begin{verbatim}
$$
	\dddnach{x}{y}{z}
$$
\end{verbatim}

\subsection{$\dznach{x}{y}$}

\begin{verbatim}
$$
	\dznach{x}{y}
$$
\end{verbatim}

\subsection{$\vnabla$}

\begin{verbatim}
$$
	\vnabla
$$
\end{verbatim}

\subsection{$\evec{x}$}

\begin{verbatim}
$$
	\evec{x}
$$
\end{verbatim}

\subsection{$\bra{x}$}

\begin{verbatim}
$$
	\bra{x}
$$
\end{verbatim}

\subsection{$\ket{x}$}

\begin{verbatim}
$$
	\ket{x}
$$
\end{verbatim}

\subsection{$\bracket{x}{y}$}

\begin{verbatim}
$$
	\bracket{x}{y}
$$
\end{verbatim}

\subsection{$\up$}

\begin{verbatim}
$$
	\up
$$
\end{verbatim}

\subsection{$\down$}

\begin{verbatim}
$$
	\down
$$
\end{verbatim}

\subsection{$\upup$}

\begin{verbatim}
$$
	\upup
$$
\end{verbatim}

\subsection{$\downdown$}

\begin{verbatim}
$$
	\downdown
$$
\end{verbatim}

\subsection{$\sandwich{x}{y}$}

\begin{verbatim}
$$
	\sandwich{x}{y}
$$
\end{verbatim}

\subsection{$\fsqrt{x}{y}$}

\begin{verbatim}
$$
	\fsqrt{x}{y}
$$
\end{verbatim}

\end{document}


