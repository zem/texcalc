\documentclass[a4paper]{article}
\usepackage[utf8]{inputenc}
\usepackage{german}
\usepackage{anfpralayout}
\usepackage{physictools}
\usepackage{texcalc}
\usepackage{amsmath}
\usepackage{hyperref}

\setlength{\parindent}{0cm}

\newcommand{\texcalc}{{\TeX}calc}
\begin{document}
\title{Anleitung zu den {\TeX}calc \LaTeXe-Packeten}
\author{Hans Freitag (und einige andere)}
\maketitle 

\texcalc Besteht aus drei Komponenten. 

\begin{enumerate}
\item texcalc.py ist ein in python geschriebener Taschenrechner der das Pythonmodul 
		uncertainties benutzt.
\item texcalc.sty ist ein \LaTeXe-Packet mit dem Berechnungen mit Messunsicherheiten 
		direkt aus dem \LaTeXe Dokument heraus möglich sind. 
\item Zusatz styles die zwar nicht direkt etwas mit \texcalc zu tun haben, aber für 
		das Schreiben von Physikprotokollen Nützlich sein können. Die beiden zusatzpackete 
		gliedern sich in:
	\begin{itemize}
	\item physictools.sty eine allgemeine toolsammlung die das Leben leichter machen soll
	\item anfpralayout.sty Layoutparameter Deckbletter etc, die wahrscheinlich nie einmal 
		irgendwo anders als an der Uniwien in der Physik verwendet werden. 
	\end{itemize}
	Die Contrib Packete sind eine Sammlung von \LaTeXe-Makros von Studierenden der Fakultät, 
	die sich im Laufe des Studiums als Hilfreich erwiesen haben. 
\end{enumerate}


\tableofcontents

\section{Features}


\section{Download}

\subsection{Abhängigkeiten}

\begin{itemize}
\item Eine LaTeX Distribution wie MikTeX oder TeXlive
\item Python 
\item Python-numpy
\item Python-uncertainties
\end{itemize}

\subsection{Dieses Packet}

\begin{itemize}
	\item Download der aktuellen Version von GitHUB \url{https://github.com/zem/texcalc/archive/master.zip}
\end{itemize}

\subsection{Software für Windows}

\begin{itemize}
	\item MikTeX \url{http://mirrors.ctan.org/systems/win32/miktex/setup/basic-miktex-2.9.4757.exe}
	\item TeXnicenter \url{http://www.texniccenter.org/resources/downloads/29-downloads/12-texniccenter-installeir}
	\item Python \url{http://www.python.org/ftp/python/3.3.0/python-3.3.0.msi}
	\item Python Numpy \url{http://sourceforge.net/projects/numpy/files/latest/download?source=files}
	\item Python Uncertainties \url{http://pypi.python.org/packages/source/u/uncertainties/uncertainties-1.9.tar.gz}
\end{itemize}

\subsection{Software für Linux(Debian)}

\begin{verbatim}
	apt-get install texlive latexila python python-numpy python-uncertainties
\end{verbatim}

(Wer will kann und soll natürlich auch vim verwenden, oder meinetwegen auch Emacs.)

\subsection{Mitentwickeln}

Mitentwickeln kann jeder. Dafür brauchst du das Versionskontrollsystem GIT
\footnote{GIT ist übrigens auch Prima dafür geeignet um LaTeX dokumente zu verwalten, 
sowohl für einen Persönlich als auch wenn mehrere Personen ein Dokument bearbeiten. } 
einfach das Repository klonen, entweder von nawi.at: 

\begin{verbatim}
	git clone http://www.nawi.at/git/texcalc
\end{verbatim}

Oder über github.com Was sich so tut kannst du im gitweb nachlesen: 
\url{http://www.nawi.at/gitweb/?p=texcalc;a=summary} oder halt auf github.com\\

Wenn du einen Patch machen willst kannst du 

\begin{verbatim}
	git format-patch origin/master
\end{verbatim}

verwenden, und den Patch an zem@nawi.at mailen oder auf github einen fork 
anlegen und patchen. 


\section{Installation}

Das Packet ist in der Datei {\bf phunivie.sty} implementiert, um das Packet zu 
verwenden muss es entweder in dem Verzeichnis liegen wo das \TeX-Dokument 
übersetzt wird, oder in eines der Suchpfade für Packete kopiert werden.\\



\subsection{Linux}

Installation auf Linux geht mit: 

\begin{verbatim}
	make install install-contrib
\end{verbatim}

\subsection{Windows}

Hoffentlich gibts bald ein install.bat script oder sowas. 

\begin{enumerate}
	\item Erstellen eines Ordners mit dem Namen der \textbf{\textbf{*.sty}} 
		Datei (hier phunivie) unter dem Installationsordner:\\

		\textbf{\emph{MiKTeX\textbackslash tex\textbackslash latex}}\\
		Normalerweise unter: \textbf{\emph{C:\textbackslash Program 
		Files (x86)\textbackslash MiKTeX 2.9\textbackslash 
		tex\textbackslash latex}}

	\item Einbinden in Miktex:\\
			Gehe auf $\boldsymbol{Start \rightarrow MiKTeX \rightarrow 
			Maintenance \rightarrow Settings \rightarrow Refresh~FNDB}$

	\item Fertig!
\end{enumerate}


\section{Benutzung}

\subsection{Erste Schritte mit \texcalc}




In der Preambel:

\begin{verbatim}
	\usepackage{phunivie}
\end{verbatim}

oder

\begin{verbatim}
	\usepackage[option]{phunivie}
\end{verbatim}

Optionen können sein: 

\begin{description}
	\item[basti] Sebastians Layout Parameter (margin$=$2.5cm)
	\item[motz] Motz's Layout Parameter
\end{description}
	

\section{Titelseite}

Für das Anfängerpraktikum ist eine Besondere Titelseite nötig, die 
kann natürlich ganz einfach gesetzt werden:

\begin{verbatim}
\semester{SoSe 2011}
%\fakulty{Fakultät für Sonstwas}
%\title{Physikalisches Praktikum\\für Lehramtskandidaten}
\experiment{1. Fnord}
%\date{abgabedatum}
\author{Hans Freitag}
\group{3}
\supervisor{someone}

\makeanfpratitle
\end{verbatim}

Die auskommentierten Parameter sind optional. Und so siehts 
aus\footnote{$\backslash$date und $\backslash$author gehen zur Zeit nur wenn der Block am 
Anfang eines Dokumentes steht}:

\semester{SoSe 2011}
%\fakulty{Fakultät für Sonstwas}
%\title{Physikalisches Praktikum\\für Lehramtskandidaten}
\experiment{1. Fnord}
%\date{\today}
%\author{Hans Freitag}
\group{3}
\supervisor{someone}
\makeanfpratitle


\section{physictools Packet}


\subsection{$23\E{10}$}

\begin{verbatim}
$$23\E{10}$$
\end{verbatim}

\subsection{\ddt}

\begin{verbatim}
$$
	\ddt
$$
\end{verbatim}

\subsection{$\dnach{x}$}

\begin{verbatim}
$$
	\dnach{x}
$$
\end{verbatim}

\subsection{$\ddnach{x}{y}$}

\begin{verbatim}
$$
	\ddnach{x}{y}
$$
\end{verbatim}


\subsection{$\dddnach{x}{y}{z}$}

\begin{verbatim}
$$
	\dddnach{x}{y}{z}
$$
\end{verbatim}

\subsection{$\dznach{x}{y}$}

\begin{verbatim}
$$
	\dznach{x}{y}
$$
\end{verbatim}

\subsection{$\vnabla$}

\begin{verbatim}
$$
	\vnabla
$$
\end{verbatim}

\subsection{$\evec{x}$}

\begin{verbatim}
$$
	\evec{x}
$$
\end{verbatim}

\subsection{$\bra{x}$}

\begin{verbatim}
$$
	\bra{x}
$$
\end{verbatim}

\subsection{$\ket{x}$}

\begin{verbatim}
$$
	\ket{x}
$$
\end{verbatim}

\subsection{$\bracket{x}{y}$}

\begin{verbatim}
$$
	\bracket{x}{y}
$$
\end{verbatim}

\subsection{$\up$}

\begin{verbatim}
$$
	\up
$$
\end{verbatim}

\subsection{$\down$}

\begin{verbatim}
$$
	\down
$$
\end{verbatim}

\subsection{$\upup$}

\begin{verbatim}
$$
	\upup
$$
\end{verbatim}

\subsection{$\downdown$}

\begin{verbatim}
$$
	\downdown
$$
\end{verbatim}

\subsection{$\sandwich{x}{y}$}

\begin{verbatim}
$$
	\sandwich{x}{y}
$$
\end{verbatim}

\subsection{$\fsqrt{x}{y}$}

\begin{verbatim}
$$
	\fsqrt{x}{y}
$$
\end{verbatim}

\end{document}


